% Created 2013-05-26 Вс. 21:54
\documentclass[presentation]{beamer}
\usepackage[utf8]{inputenc}
\usepackage[T1]{fontenc}
\usepackage{fixltx2e}
\usepackage{graphicx}
\usepackage{longtable}
\usepackage{float}
\usepackage{wrapfig}
\usepackage{soul}
\usepackage{textcomp}
\usepackage{marvosym}
\usepackage{wasysym}
\usepackage{latexsym}
\usepackage{amssymb}
\usepackage{hyperref}
\tolerance=1000
\usepackage[english,russian]{babel}
\usepackage{graphicx}
\usepackage{amsfonts}
\usepackage{color}
\usepackage{algorithmic} \usepackage[ruled]{algorithm}
\usetheme{Warsaw}
\usepackage{concrete}
\centering
\def\A{\mathbf{A}}
\def\V{\mathbf{V}}
\def\B{\mathbf{B}}
\def\C{\mathbf{C}}
\providecommand{\alert}[1]{\textbf{#1}}

\title{Исследование и развитие технологий автогенерации кода для GPU в применении к задачам тензорной аппроксимации. Алгоритм ALS}
\author{Кузнецов М.А.}
\date{20 мая 2013}
\hypersetup{
  pdfkeywords={},
  pdfsubject={},
  pdfcreator={Emacs Org-mode version 7.8.02}}

\begin{document}

\maketitle



\section{Cодержание}
\label{sec-1}
\begin{frame}
\frametitle{Утверждения}
\label{sec-1-1}

\begin{itemize}
\item Тензорные алгоритмы сложные и требуют больших вычислительных затрат
\item GPU обладают большой мощностью в плане вычислений
\end{itemize}
Отсюда логичный вывод --- попытаться писать код алгоритма для GPU
\end{frame}
\begin{frame}
\frametitle{Код для GPU --- особенности}
\label{sec-1-2}

У программирования для GPU есть свои плюсы и минусы

``+''
\begin{itemize}
\item Возможность использования множества процессоров
\item Возможность серьезного ускорения программы
\end{itemize}
``-''
\begin{itemize}
\item Очень длительный процесс написания кода
\item Сложность написания кода
\item Платформенно-зависим
\end{itemize}
\end{frame}
\begin{frame}[fragile]
\frametitle{Пример кода для GPU}
\label{sec-1-3}


\begin{verbatim}
__kernel void __attribute__ ((reqd_work_group_size(16, 1, 1))) loopy_kernel(__global float const *restrict a, __global float const *restrict v, __global float const *restrict w, __global float *restrict f, long const n, long const r)
{
  float acc_j_outer_j_inner_k_outer_k_inner;

  if ((-1 + -16 * gid(0) + -1 * lid(0) + n) >= 0)
  {
    acc_j_outer_j_inner_k_outer_k_inner = 0.0f;
    for (int j_inner = 0; j_inner <= 15; ++j_inner)
      for (int k_inner = 0; k_inner <= 15; ++k_inner)
        for (int k_outer = 0; k_outer <= (-1 + -1 * k_inner + ((15 + n + 15 * k_inner) / 16)); ++k_outer)
          for (int j_outer = 0; j_outer <= (-1 + -1 * j_inner + ((15 + n + 15 * j_inner) / 16)); ++j_outer)
            acc_j_outer_j_inner_k_outer_k_inner = acc_j_outer_j_inner_k_outer_k_inner + a[n * n * (lid(0) + gid(0) * 16) + n * (j_inner + j_outer * 16) + k_inner + k_outer * 16] * v[n * (lid(1) + gid(1)) + j_inner + j_outer * 16] * w[n * (lid(1) + gid(1)) + k_inner + k_outer * 16];
    f[n * (lid(1) + gid(1)) + lid(0) + gid(0) * 16] = acc_j_outer_j_inner_k_outer_k_inner;
  }
}
\end{verbatim}
\end{frame}
\begin{frame}
\frametitle{Идея автоматической генерации}
\label{sec-1-4}

Код на OpenCL сложный, нужно учитывать барьеры, разбиение и вычислительные сетки.
Поэтому логично использвать пакеты автоматической генерации кода. 
Использован новый, активно разрабатывающийся пакет loo.py
\end{frame}
\begin{frame}
\frametitle{Почему Python}
\label{sec-1-5}

Вообще говоря, Python медленный язык. Почему же его можно использовать для ускорения работы программ?
Однако:
\begin{itemize}
\item Python обладает большим количеством стандартных модулей
\item Использование модулей дает возможность получить скорость C-кода
\item Простой синтаксис языка позволяет быстро писать программы даже для сложных алгоритмов
\item Экономит время программиста
\end{itemize}
\end{frame}
\begin{frame}
\frametitle{Пакет Loo.py}
\label{sec-1-6}

Пакет loo.py разрабатывается Андреасом Клекнером (Andreas Kloekner).
\href{http://gitlab.tiker.net/inducer/loopy}{http://gitlab.tiker.net/inducer/loopy} 
Вышеуказанный репозиторий --- закрытый, однако автор включен в список ``разработчиков''.
Стабильную версию можно найти здесь \href{http://git.tiker.net/loopy.git}{http://git.tiker.net/loopy.git}

Пакет предназначен для ``развертки'' циклов и последующей генерации OpenCL кода.
Для этого нужно сформулировать ядро в специальном синтаксисе.
\end{frame}
\begin{frame}[fragile]
\frametitle{Как правильно сформулировать алгоритм}
\label{sec-1-7}

Для применения пакета loo.py алгоритм должен быть формализован и записан в виде
вложенных циклов:

\begin{verbatim}
for i in xrange(...):
     for j in xrange(....):
...............
          for k n xrange(....):
           .............
\end{verbatim}
\end{frame}
\begin{frame}[fragile]
\frametitle{Пример ядра}
\label{sec-1-8}


\begin{verbatim}
def Prav_U(ctx):
  order='C'
  dtype = np.float32
  knl = lp.make_kernel(ctx.devices[0], 
  [

    "{[i,j,k,alpha]: 0<=alpha<r and 0<=i,j,k<n}",

  ],
  [
    "f[alpha,i]=sum((j,k), a[i,j,k]*v[alpha,j]*w[alpha,k])",
  ],
  [
    lp.GlobalArg("a", dtype, shape="n, n, n", order=order),
\end{verbatim}
\end{frame}
\begin{frame}[fragile]


\begin{verbatim}
lp.GlobalArg("v", dtype, shape="r, n", order=order),
    lp.GlobalArg("w", dtype, shape="r, n", order=order),
    lp.GlobalArg("f", dtype, shape="r, n", order=order),
    lp.ValueArg("n", np.int64),
    lp.ValueArg("r", np.int64),
  ],
  assumptions="n>=1")
  knl = lp.split_iname(knl, "i", 16,outer_tag="g.0", inner_tag="l.0")
  knl = lp.split_iname(knl, "alpha", 1, outer_tag="g.1", inner_tag="l.1")
  knl = lp.split_iname(knl, "j", 16)
  knl = lp.split_iname(knl, "k", 16)
  print lp.CompiledKernel(ctx, knl).get_highlighted_code()   
  return knl
\end{verbatim}
\end{frame}
\begin{frame}
\frametitle{Актуальность исследования}
\label{sec-1-10}

Привлекательность исследования обусловлена несколькими факторами:
\begin{enumerate}
\item Тензорные алгоритмы начали активно разрабатываться в последнее время
\item Написание GPU-кода --- сложная задача, существует необходимость исследовать возможности автогенерации GPU-кода
\item Вычислительная мощность GPU превосходит многоядерные CPU, использование GPU эффективней
\end{enumerate}
\end{frame}
\begin{frame}

Ввиду того, что процесс написания GPU-кода вручную длительный и трудоемкий, хоть и эффективный,
 в вычислительных задачах хотелось бы использовать следующий ``идеальный'' способ его написания:
\begin{enumerate}
\item Использование в динамических языках (Python)
\item Автоматическое распараллеливание стандартных задач (циклов), генерация OpenCL/CUDA-кода
\item Быстрый процесс написания кода
\end{enumerate}
\end{frame}
\begin{frame}
\frametitle{Цель работы}
\label{sec-1-12}

\begin{itemize}
\item Научиться использовать пакет loo.py
\item С помощью пакета получить эффективную параллельную реализацию алгоритма ALS
\item Научиться ``параллелить'' тензорные алгоитмы
\end{itemize}
\end{frame}
\begin{frame}
\frametitle{Метод ALS: идея}
\label{sec-1-13}


Основная идея алгоритма, состоит в том, чтобы фиксировать все факторы,
кроме одного, канонического разложения и искать минимум функционала 
\begin{equation*}
F=\sum_{i,j,k=1} (A_{ijk}-\sum_{\alpha=1}^r U_{i\alpha}V_{j\alpha}W_{k\alpha})^2.
\end{equation*}
только по нему.
Путем циклических перестановок, используя уже полученные факторы, строятся последующие, до тех пор,
пока не будет достигнута требуемая точность аппроксимации или, пока не сработают другие критерии
остановки алгоритма
\end{frame}
\begin{frame}
\frametitle{Формулы метода ALS}
\label{sec-1-14}


Найдем частную производную функционала F по U$_{\hat i\hat\alpha}$ и приравняем ее к 0:
\begin{equation*}
\frac{\partial F}{\partial U_{\hat i \hat \alpha}} = 
2 \Big( \sum_{i,j,k} (A_{ijk}-\sum_{\alpha} U_{i \alpha}V_{j\alpha}W_{k\alpha})\Big)\Big(-
\sum_{\check \alpha}\ (V_{j\check \alpha}W_{k\check \alpha})
\frac{\partial U_{i \check \alpha}}{\partial U_{\hat i \hat \alpha}}\Big) =0;
\end{equation*}
\begin{equation*}
\frac{\partial U_{i \check \alpha}}{\partial U_{\hat i \hat \alpha}} =
\delta_{i,\hat i}\delta_{\check \alpha \hat \alpha};
\end{equation*}

Окончательно, получаем следующие соотношения:
\begin{equation*}
\sum_{j,k} A_{\hat ijk}V_{j \hat \alpha}W_{k\hat \alpha}=
\sum_{j,k,\alpha} U_{\hat i\alpha}V_{j\alpha}W_{k\alpha}V_{j\hat \alpha}
W_{k,\hat \alpha},
\end{equation*}
\end{frame}
\begin{frame}
\frametitle{Формулы ALS}
\label{sec-1-15}

Обозначим через M$_{\alpha \hat \alpha}$
матрицу с элементами
\begin{equation}\label{curs:lev}
M_{\alpha,\hat \alpha} = (\sum_{j}V_{j,\alpha}
V_{j\hat \alpha}) (\sum_{k}W_{k\alpha}W_{k\hat \alpha});
\end{equation}
тогда
\begin{equation}\label{curs:prav}
\sum_{\alpha} U_{\hat i, \alpha}M_{\alpha,\hat \alpha} = 
\sum_{j,k} A_{\hat i,j,k}V_{j, \hat \alpha}W_{k,\hat \alpha};
\end{equation}
Через  F$_{\mathrm{i,\hat \alpha}}$ обозначим правую часть. Тогда, имеем
\begin{equation}
\sum_{\alpha} U_{\hat i \alpha}M_{\alpha \hat \alpha}=F_{i \hat \alpha}.
\end{equation}
\end{frame}
\begin{frame}
\frametitle{В виде системы}
\label{sec-1-16}

или в виде системы линейных уравнений

\begin{equation}\label{curs:q5}
U M = F.
\end{equation}
где $M \in \mathbb{R}^{r \times r}$. 
\end{frame}
\section{Используемые платформы}
\label{sec-2}
\begin{frame}

В ходе экспериментов использовались следующие платформы:
\begin{itemize}
\item Мобильная видеокарта NVIDIA
\item Мобильный процессор Intel Core i5
\item Кластер Tesla ИВМ РАН
\end{itemize}
\end{frame}
\begin{frame}
\frametitle{Характеристики Tesla}
\label{sec-2-2}


\begin{center}
\begin{tabular}{lll}
 Device Tesla C2070                &                      &     \\
 NAME:                             &  Tesla C2070         &     \\
 VENDOR:                           &  NVIDIA Corporation  &     \\
 VERSION:                          &  304.54              &     \\
 VERSION:                          &  OpenCL 1.1 CUDA     &     \\
 OPENCL\_{} C\_{} VERSION:         &  OpenCL C 1.1        &     \\
 MAX\_{} WORK\_{} GROUP\_{} SIZE:  &  1024                &     \\
 ADDRESS\_{} BITS:                 &  32                  &     \\
 MAX\_{} MEM\_{} ALLOC\_{} SIZE:   &  1343 MByte          &     \\
 GLOBAL\_{} MEM\_{} SIZE:          &  5375 MByte          &     \\
\end{tabular}
\end{center}
\end{frame}
\section{Численные эксперименты}
\label{sec-3}
\begin{frame}
\frametitle{Таблицы времен}
\label{sec-3-1}

Для фиксированного ранга $r=3$ и размерности тензора $n$ исследована скорость выполнения как отдельных ядер, 
так и всего алгоритма ALS. Однако ALS алгоритм не гарантирует сходимость, только
убывание невязки, поэтому будем указывать только время выполнения одной итерации.
Приведем таблицу с временем выполнения.

\begin{center}
\begin{tabular}{lrrrr}
 размер n              &         128  &         256  &       512  &       756  \\
 t$_r$                 &    0.013803  &     0.08674  &   0.65225  &   0.92513  \\
 t$_l$                 &  0.00035595  &   0.0004210  &  0.000552  &  0.000673  \\
 t$_{\mathrm{solve}}$  &  0.00025391  &  0.00025510  &  0.000256  &  0.000256  \\
 LU                    &  0.00024890  &   0.0002851  &   0.00035  &  0.000391  \\
 T$_i$                 &    0.026740  &      0.1834  &   1.08289  &   1.92985  \\
\end{tabular}
\end{center}
\end{frame}
\begin{frame}

Приведем также таблицу с временем выполнения одной итерации программы, вычисления
правой части в зависимости от ранга $r$ и фиксированной размерности тензора $n=128$

\begin{center}
\begin{tabular}{lrrrr}
 ранг r  &        3  &       6  &      10  &      20  \\
 t$_r$   &  0.01380  &  0.0152  &  0.0162  &  0.0184  \\
 T$_i$   &  0.04326  &  0.0437  &  0.0468  &  0.0556  \\
\end{tabular}
\end{center}
\end{frame}
\begin{frame}
\frametitle{График}
\label{sec-3-3}

Для наглядности также построим графики поведения времени вычисления правой части
на CPU, мобильном GPU и Tesla:

\begin{center}
\begin{figure}[H]
\centering
\includegraphics[width=8cm]{Illustration.jpg}
\caption{Зависимость времени выполнения одной итерации от размера $n$.На графике синяя линия соответствует мобильному GPU, зеленая CPU, красная Tesla. Обрывы линий означают, что тензор большего размера уже не помещается в память.}
\end{figure}
\end{center}
\end{frame}
\section{Выводы и планы}
\label{sec-4}

В ходе выполениния работы были получены следующие результаты:
\begin{enumerate}
\item Изучен пакет автоматической генерации OpenCL-кода
\item Реализованы алгоритмы:
\begin{itemize}
\item LU-разложения, решения систем в стандартном виде LU
\item подсчета правой части алгоритма ALS
\item ALS-алгоритм
\end{itemize}
\end{enumerate}
\begin{frame}
\frametitle{Важные выводы}
\label{sec-4-1}

Ключевые выводы:
\begin{enumerate}
\item Генерировать OpenCL код можно автоматически
\item Сильно экономится время, а качество реализации не страдает
\item Можно избежать ошибок ``технического'' характера
\item Можно параллелить произвольный алгоритм, записанный в нужном формате
\end{enumerate}
\end{frame}
\begin{frame}
\frametitle{Планы}
\label{sec-4-2}

\begin{itemize}
\item Оптимизировать имеющийся код
\item На основе имеющегося опыта распараллелить другие тензорные алгоритмы
\item Написать небольшую статью совместно с Андреасом Клекнером
\end{itemize}
\end{frame}
\begin{frame}
\frametitle{Вопросы}
\label{sec-4-3}

Спасибо за внимание!
Ваши вопросы?
\end{frame}

\end{document}